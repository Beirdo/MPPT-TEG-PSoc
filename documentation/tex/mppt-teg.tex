\documentclass{report}

\def\glossaryname{Glossary}
\def\theglossary{\chapter{\glossaryname} \begin{infomap}}
\def\endtheglossary{\end{infomap}}

\makeglossary

\begin{document}
\pagenumbering{roman}
\title {Design documentation for MPPT-TEG design}
\author {Gavin Hurlbut \\ Originally written starting September 17, 2019}
\maketitle
\tableofcontents
\eject
\listoffigures
\listoftables
\eject

\setcounter{chapter}{0}
\renewcommand\chaptername{\space}
\renewcommand\thechapter{}

\chapter{Introduction}
I have for a while now been interested in alternate energy sources.  This 
comes to head particulary when I'm being ``off the grid'' for a time, which
usually means while camping.  I like to get away from it all and go camping, but
being as how I am a geek at nature, I usually end up bringing electronics like
my cell phone and even my laptop with me.  Coding up a pet project while lying
in a hammock enjoying nature is very tempting.  The problem is:  I end up
running out of battery power far too fast.

While stumbling around the web looking for trouble... er ideas... I stumbled
across a site where this enterprising dude made a hot water system that
operated with no power, using the heat of the campfire and convection (and 
siphoning) to circulate water through a coil of copper tubing that he built
his campfire on top of.  He had spectacular results, heating a large cooler 
full of water nearly to a boil.

I was also reading about how using a thermo-electric cooling (TEC) device in
reverse as a thermo-electric generator (TEG) was possible.  This is done using
the corollary of the Peltier effect (which cools one side of a bi-metal 
junction and heats the other when electricity is applied), which is known as 
the Seebeck effect.  The Seebeck effect is described as "a phenomenon in which
a temperature difference between two dissimilar electrical conductors or
semiconductors produces a voltage difference between the two substances." \cite{WI}.

This set my engineering mind in motion, and I began to design a power generation
unit that would use the hot water heated by the campfire (or other means if
necessary), and use it to generate power by presenting a (hopefully) large
temperature differential across TEG units.  To do this, I would need to have
another closed-circuit water flow that extracts heat from the hot water ``tank''
source reservoir using an immersion wort chiller coil (from homebrewing), and
pumping it through watercooled heat-exchange blocks that were directly attachedi
to the ``hot'' side of the TEG, and then recycled back into another reservoir,
where the input of the wort chiller can then cycle it through again.  Thei
``cold'' side of the TEG is attached to a large heatsink with air blowing
across it.  For safety to keep the TEG from overflowing, I plan to use a 
solenoid-controlled valve to divert the hot water right back into the reservoir
rather than passing it through the heat exchangers on the TEG units.  I also
have the pump controlled by the CPU so it can be shut right off.

From the multiple TEG units, I want to charge lead-acid batteries (either 
deep-cycle batteries or sealed gel-cells) and also lithium-ion batteries (18650
format) that would be used for bootstrapping the system.  This requires a common
bus that all cells are powering, which might cause me issues with load-sharing,
but we shall see.

I also decided (as I am an engineer after all) that I needed copious monitoring,
including many temperature sensors, and also, if possible the flow rate of the
circulating water, and also voltages, currents, etc all over the place.  And,
of course, I want to have an LCD screen on there to display the measurements
and a WiFi connection to do an IoT-type connection via my cellphone.

I found a nice "Thermo-electric cooling" device on Amazon that I should be able
to leverage with minimal work to work in this way.  It has 8 TEG units
installed (4 on each side of the square heatsink with a fan blowing through
the middle.

As I am used to using Atmel MCUs (AVR, especially) I was thinking of using
Arduino for the control, but I remembered that I was recently introduced to 
the Cypress PSoC series, with the newer ones using ARM Cortex M-series CPU
cores.  Challenge accepted.  My design is based around two PSoC devices: a PSoC
4L series one to do the MPPT calculations and generate the corresponding PWM
control waveforms and phased clock for the second DC-DC converter, and a PSoC
5LP device that wrangles the large pile of data and publishes it in CBOR format
to a custom IoT server, and also writes it to an SD card, and has a touchscreen
LCD GUI.

\eject
\setcounter{page}{1}
\pagestyle{headings}
\pagenumbering{arabic}

\setcounter{chapter}{0}
\renewcommand\chaptername{Chapter}
\renewcommand\thechapter{\arabic{chapter}}

\chapter{DC-DC Design Calculations}
\section{MPPT DC-DC converter}

The first DC-DC converter is to extract the most power possible fromt the TEG
module.  This power point is at a voltage half-way between $V_{short}$ and
$V_{open}$, and a current half-way between $I_{open}$ and $I_{short}$.  As
$V_{short} = 0V$ and $I_{open} = 0A$, this value can be easily estimated via
linear regression as the V/I relationship at any operating point (temperature
differential) is a linear slope down from an Y-intercept of $V_{open}$ down to
an X-intercept of $I_{short}$.  Given any two points along that line, we can
calculate the two intercepts, and thus calculate the expected MPPT point.

This DC-DC converter is a boost converter that aims to convert to a maximum
voltage of 18V, but will do so by controlling the current flow to maximize the
power.  This is not the normal mode used in a boost controller, and to protect
the output filter capacitors, an 18V zener diode is in place to ensure the 
voltage cannot swing too high.  As we are aiming for a ``constant'' V/I poin t
depending on the thermal difference across the TEG, and also on its
characteristics, the output voltage may vary.

\section{Output DC-DC Converter}

To be able to harness all of the TEG units at the same time, the voltage
output of each must be at the same level.  Even this isn't simple to arrange
as the normal ``simple'' method to gang supplies together is to feed into the
common bus using diodes, and the tolerance in the forward voltage drop in the
diodes makes it unlikely that the multiple feeds will be at the exact same
voltage which will cause some sources to take ``full'' load, and some to take
nearly none.  To mitigate this, I will be taking the feedback for the second
DC-DC converter directly from the output bus (after diodes) if possible.

This output DC-DC converter (per TEG channel) is another boost converter.  This time the input/output characteristics are more the normal case for a converter,
so it was easier to model.  Essentially, we will be converting 9-18V input to a
regulated 24V output, and then tying all of the 24V outputs together onto a 
24V bus.  The battery chargers (and local power supplies) will be regulated
down off the 24V bus primarily.

\section{+12V DC-DC Converter}

The +12V supply is used for 3 external things: the fans on the heat sink, the
water pump, and the solenoid on the dump valve.  As the primary purpose of this
design is to charge 12V lead-acid batteries, I am feeding the 12V rail (via
diodes) from the 12V DC-DC converter, and from both attached 12V lead-acid
batteries.  The idea is that once we are charging, the batteries should be at
around 13.2V and will thus be sourcing all of the load on the 12V rail.

\section{+3.3V DC-DC Converter}

The +3.3V supply runs nearly all of the electronics in the design.  The DC-DC
converter is sourced (via diodes) from the lithium-ion battery or from the +24V
rail.  This allows for a bootstrapping of the system to get it up to speed
before any significant power can be generated, and also allows it to shutdown
nicely.  Once the +24V rail exceeds the LiIon battery (about 4.6V) the
system electronic load should be coming from the +24V rail.

\section{+5V DC-DC Converter}

The +5V supply is also sourced in the same way as the +3.3V rail.  Not much
of the design uses +5V (mostly in the battery charging bits).

\eject
\addcontentsline{toc}{chapter}{Bibliography}
\nocite{*}
\bibliography{\jobname}
\bibliographystyle{alpha}
\end{document}
